\section*{Aufgabe 5: }
Seien $n, m \in \mathbb{N}$, $P$ eine Menge von Punkten mit $|P| = n$, wobei für alle $P_i \in P$ gilt, dass $P_i \in \mathbb{R}^m$ und $\{L_1^n, \hdots, L_n^n\}$ eine Menge von Basisfunktionen. Weiter sei $T(P_i) = L(P_i) + t$ eine affine Transformation, wobei $L$ eine lineare Transformation und $t \in \mathbb{R}^m$ ist.\\
Zu zeigen:
$$
T\left(\sum_{i=0}^{n}P_iL_i^n(u)\right) = \sum_{i=0}^{n}T\left(P_iL_i^n(u)\right)
$$
gdw.
$$
\sum_{i=0}^{n}L_i^n(u) = 1 
$$
Sei $l = \sum_{i=0}^{n}L_i^n(u)$. Aufgrund der Linearität von L folgt:
$$
T\left(\sum_{i=0}^{n}P_iL_i^n(u)\right) = L\left(\sum_{i=0}^{n}P_iL_i^n(u)\right) + t = t + \sum_{i=0}^{n}L(P_i)L_i^n(u)
$$
und
$$
\sum_{i=0}^{n}T(P_i)L_i^n(u) = \sum_{i=0}^{n}(LP_i + t)L_i^n(u) = \sum_{i=0}^{n}L(P_i)L_i^n(u) + \sum_{i=0}^{n}tL_i^n(u) = lt + \sum_{i=0}^{n}L(P_i)L_i^n(u)
$$
Offensichtlich sind die beiden Terme nur für $l = 1$ identisch. Damit ist gezeigt, was zu zeigen war.