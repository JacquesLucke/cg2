\section*{Aufgabe 4: }
Gegeben sind die fünf Kontrollpunkte $P_0 = (0, 0)$, $P_1 = (1, 3)$, $P_2 = (2, 1)$, $P_3 = (3, 2)$ und $P_4 = (4, 1)$.
Der Knotenvektor $T$ des B-Splines ersten Grades hat damit $m = n + k + 1 = 5 + 1 + 1 = 7$ Elemente. Mit $T = (0, 1, 2, 3, 4, 5, 6)$ folgt dann für die Basisfunktionen des B-Splines:
\begin{align*}
  N_0^1(t) &= \frac{t - t_0}{t_1 - t_0}\cdot N_0^0(t) + \frac{t_2 - t}{t_2 - t_1}\cdot N_1^0(t) = t \cdot N_0^0(t) + (2 - t) \cdot N_0^0(t)\\
  N_1^1(t) &= \frac{t - t_1}{t_2 - t_1}\cdot N_1^0(t) + \frac{t_3 - t}{t_3 - t_2}\cdot N_2^0(t) = (t - 1) \cdot N_1^0(t) + (3 - t) \cdot N_2^0(t)\\
  N_2^1(t) &= \frac{t - t_2}{t_3 - t_2}\cdot N_2^0(t) + \frac{t_4 - t}{t_4 - t_3}\cdot N_3^0(t) = (t - 2) \cdot N_2^0(t) + (4 - t) \cdot N_3^0(t)\\
  N_3^1(t) &= \frac{t - t_3}{t_4 - t_3}\cdot N_3^0(t) + \frac{t_5 - t}{t_5 - t_4}\cdot N_4^0(t) = (t - 3) \cdot N_3^0(t) + (5 - t) \cdot N_4^0(t)\\
  N_4^1(t) &= \frac{t - t_4}{t_5 - t_4}\cdot N_3^0(t) + \frac{t_6 - t}{t_6 - t_5}\cdot N_5^0(t) = (t - 4) \cdot N_3^0(t) + (6 - t) \cdot N_5^0(t)\\
\end{align*}
Diese Basisfunktionen resultieren in folgendem Graphen:
\begin{center}
  \begin{figure}[H]
    \includegraphics[width=\textwidth, keepaspectratio]{splines}
  \end{figure}
\end{center}
[TODO: Erklärung Verbindung Grad, Stetigkeit, (lokaler) Support]