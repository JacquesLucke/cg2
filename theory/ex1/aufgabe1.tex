\section*{Aufgabe 1: }
Sei $N = \{n_1 \cdots n_n\}$ eine Menge von Zahlen. Eine Möglichkeit, den Median dieser Zahlen zu finden, ist der Quickselect-Algorithmus. Dieser erhält als Eingabe ein Array aus Zahlen $N$ und eine Zahl $k\in [0, |N|]$ und findet die k-kleinste Zahl aus $N$. Damit lässt sich der Median durch folgende einfache Fallunterscheidung finden:
\begin{center}
  \begin{figure}[H]
    \begin{itemize}
    \item Falls $|N| \mod 2 = 1$ gilt, hat der Median den Wert $qs(N, \lfloor \frac{N}{2} \rfloor)$
    \item Ansonsten hat der Median den Wert $\frac{qs(N, \frac{N}{2}-1) + qs(N, \frac{N}{2})}{2}$
    \end{itemize}
  \end{figure}
\end{center}
Quickselect findet das k-kleinste Element, indem ähnlich wie bei Quicksort ein Pivotelement $p$ ausgewählt und die Umgebung des Pivotelements in Linearzeit so umsortiert wird, dass danach alle Elemente, die kleiner als $p$ sind links von $p$ und alle anderen rechts von $p$ stehen. Steht $p$ danach an $k$-ter Stelle im Array, ist $p$ die gesuchte Zahl. Andernfalls muss ein neues Pivotelement ausgewählt und die unsortierte Umgebung des neuen Pivotelementes erneut partitioniert werden, bis das passende Pivotelement gefunden wurde.\\\\
Die Laufzeit von Quickselect ist im Averagecase $\mathcal{O}(n)$, im Worstcase (bei einer schlechten Auswahl des Pivotelements) allerdings $\mathcal{O}(n^2)$. Für diesen Fall kann der ``Median of Medians''-Algorithmus verwendet werden, um das $k$-kleinste Element zu schätzen und so ein besseres Pivotelement zu wählen.
Dieser verbesserte Algorithmus heißt Introselect und besitzt eine Worstcase-Laufzeit von $\mathcal{O}(n)$.