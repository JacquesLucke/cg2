\section*{Aufgabe 1: }
Sei $F:\mathbb{R}^3\rightarrow\mathbb{R}$ eine stetig differenzierbare Funktion mit $F(x_0, y_0, z_0) = 0$.
Dann existieren nach dem Satz von der impliziten Funktion eine Menge $U \subseteq \mathbb{R}^2$ und eine eindeutige Funktion $g:U\rightarrow\mathbb{R}$ mit
\begin{align*}
  g(x_0, y_0) &= z_0\\
  F(\vec{x}, g(\vec{x})) &= 0 \text{, }\forall \vec{x} \in U
\end{align*}
Offensichtlich wird durch den letzten Ausdruck eine implizite Oberfläche beschrieben. Folglich muss eine Funktion $F:\mathbb{R}^3\rightarrow\mathbb{R}$ stetig differenzierbar sein und in mindestens einem Punkt den Wert 0 annehmen, um implizit eine Oberfläche zu definieren.