\section*{Aufgabe 2: }
Seien die beiden Kreise $C_1$ und $C_2$ mit den Radien $r_1, r_2 \in \mathbb{R}$ durch Kegelschnitte gegeben.
Dann werden diese durch folgende Funktionen beschrieben:
\begin{align*}
  g_1(x, y, w) &= \left(\begin{matrix}x &y& w\end{matrix}\right)\left(\begin{matrix}1&0&0\\0&1&0\\0&0&-{r_1}^2\end{matrix}\right)\left(\begin{matrix}x\\y\\w\end{matrix}\right)\\
  g_2(x, y, w) &= \left(\begin{matrix}x &y& w\end{matrix}\right)\left(\begin{matrix}1&0&0\\0&1&0\\0&0&-{r_2}^2\end{matrix}\right)\left(\begin{matrix}x\\y\\w\end{matrix}\right)\\
\end{align*}
Für die algebraische Summe der beiden Kreise gilt dann:
\begin{align*}
  g(x, y, w) &= \left(\begin{matrix}x &y& w\end{matrix}\right)\left(\begin{matrix}1&0&0\\0&1&0\\0&0&-{r_1}^2\end{matrix}\right)\left(\begin{matrix}x\\y\\w\end{matrix}\right) +
  \left(\begin{matrix}x &y& w\end{matrix}\right)\left(\begin{matrix}1&0&0\\0&1&0\\0&0&-{r_2}^2\end{matrix}\right)\left(\begin{matrix}x\\y\\w\end{matrix}\right)\\
             &= \left(\begin{matrix}x &y& w\end{matrix}\right)\left(\begin{matrix}2&0&0\\0&2&0\\0&0&-({r_1}^2+{r_2}^2)\end{matrix}\right)\left(\begin{matrix}x\\y\\w\end{matrix}\right)\\
             &= \left(\begin{matrix}x &y& w\end{matrix}\right)\left(\begin{matrix}1&0&0\\0&1&0\\0&0&-\frac{1}{2}({r_1}^2+{r_2}^2)\end{matrix}\right)\left(\begin{matrix}x\\y\\w\end{matrix}\right)\\
\end{align*}
Damit ist die Summe von zwei durch Kegelschnitte beschriebenen Kreise mit den Radien $r_1$ und $r_2$ ein weiterer Kreis mit dem Radius $\sqrt{\frac{1}{2}({r_1}^2+{r_2}^2)}$.
